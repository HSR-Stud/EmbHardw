\section{Parallelism \weekDoran{4}}
	\subsection{Coprocessors \weekPageDoran{4}{6-7}}
	
		Classical coprocessors have their own instruction set. The use of the device is triggered by co-processor instructions in the programme code. The co-processor generally has some form of exception registration.
		
		\vspace{0.5cm}
		
		\textbf{Example mechanism:}
		
		\begin{enumerate}
		  \item Processor receives unidentified instruction.
		  \item Processor asks its co-processors whether this instruction applies to them.
		  \begin{enumerate}
			  \item If it doesn't, the processor raises an unidentified op-code exception and let normal processing take over.
			  \item If it does, the instruction is passed to the co-processor which then executes it.
			  \item If the co-processor is busy while the processor has a co-processor instruction lined up, the processor will stall.
		  \end{enumerate}	  
		  \item The processor will continue to execute instructions while the co-processor does the same with the passed instruction.
		\end{enumerate}


		\begin{table}[H]
			\centering
			\begin{tabular}{|p{0.425\textwidth}|p{0.425\textwidth}|}
				\hline
				\textbf{Loosely Coupled Co-Processors}
					& \textbf{Loosely Coupled Co-Processors}\\
				\hline
				Access to memory and external resources via system bus.
					& Access to memory and external resources via specialised interface.\\
				\hline
			\end{tabular}
		\end{table}
		
	\subsection{Pipelines \weekPageDoran{4}{8-29}}
	
		\begin{tabular}{cc}
			\begin{minipage}{0.475\textwidth}
				\begin{figure}[H]\centering
					\includegraphics[width=\textwidth]{./pictures/pipelining.png}
				\end{figure}
			\end{minipage}
			&	
				\begin{minipage}{0.475\textwidth}
					\begin{table}[H]
						\centering
						\begin{tabular}{|>{\bfseries}l|l|}
							\hline
							\multicolumn{2}{|c|}{\textbf{Pipeline Legend}}\\
							\hline
							FI
								& Fetch instruction\\
							\hline
							DI
								& Decode instruction\\
							\hline
							CO
								& Calculate operand address\\
							\hline
							FO
								& Fetch operand\\
							\hline
							EI
								& Execute instruction\\
							\hline
							WO
								& Write operand\\
							\hline
						\end{tabular}
					\end{table}		
				\end{minipage}\\
		\end{tabular}
		
		\subsubsection{CISC vs. RISC Architecture}
		\begin{table}[H]
			\centering
			\begin{tabular}{|p{0.425\textwidth}|p{0.425\textwidth}|}
				\hline
				\textbf{RISC}
					& \textbf{CISC}\\
				\hline
				Reduced Instruction Set Computing
					& Complex Instruction Set Computing\\
				\hline
				Each instruction is divided in different sub-instructions. These sub-instructions are kept separately, therefore the circuitry is much simpler.
					& Each instruction is actually a series of operations. E.g. loading input, performing the operation and storing the result.\\
				\hline
				Each instruction can be completed in one clock cycle.
					& Execution time of an instruction depends largely on where data is stored and many other factors.\\
				\hline
				Fixed instruction length and format.
					& Variable instruction length and format.\\
				\hline
			\end{tabular}
		\end{table}		
		
		For CISC-processors, separate hardware blocks are used for each instruction. Therefore it is possible to assign a hardware block for each sub-instruction which does its job and is then idle until the same sub-instruction of the next instruction comes along. Thus, this architecture allows for pipelining.
		
		\subsubsection{Hazards}
			\begin{longtable}{|>{\bfseries}p{0.09\textwidth}|p{0.4\textwidth}|p{0.46\textwidth}|}
				\hline
				Structural Hazards
					& \textbf{Occurs because} resource is requested by more than one instruction at the same time.\newline
						\textbf{E.g.:} Memory cannot accept another transfer within a clock cycle.\newline
						\textbf{Possible resolutions:} Duplicate resources, pipeline functional units too (e.g. ALU, Floating point unit..)
						
					& \vspace{0pt}
					
						\includegraphics[width=0.45\textwidth]{./pictures/structuralHazard.png}\\
				\hline
				Data\newline Hazards
					& \textbf{Occurs because} if next instruction depends on result of previous instruction.\newline 
						\textbf{Possible resolutions:} Using a technique called forwarding/bypassing: Result of previous instruction will be forwarded directly to where the next instruction needs it, instead of stalling until the previous instruction writes back the result.
					& \vspace{0pt}
					
						\includegraphics[width=0.45\textwidth]{./pictures/dataHazard.png}\\
				\hline
				Control Hazards
					& \textbf{Occurs because} of branch instructions .\newline
						\textbf{E.g.:} It is unknown that a branch instruction was fetched before it is decoded. At this point the instructions that followed the branch instruction in memory were already fetched and decoded, but these instructions are not executed.\newline
						\textbf{Possible resolutions:} Several approaches, e.g. branch prediction(use statistics to predict branch outcome)
					& \vspace{0pt}
					
						\includegraphics[width=0.45\textwidth]{./pictures/controlHazard.png}\\
				\hline
			\end{longtable}
		
		\begin{minipage}[t][7cm]{0.475\textwidth}
			\subsubsection{Superpipelining}
			Superpipelining is the division of stages in substages, and thus increases the number of instructions that are supported at a given moment. By dividing each stage into two substages, the clock cycle period will be halved.
			\vfill
			\begin{figure}[H]\centering
				\includegraphics[width=\textwidth]{./pictures/superpipelined.png}
			\end{figure}
		\end{minipage}
		\hfill
		\begin{minipage}[t][7cm]{0.475\textwidth}
			\subsubsection{Superscalar}
			Superscalar architectures allow multiple instructions per clock cycle by running multiple pipelines in parallel.
			\vfill
			\begin{figure}[H]\centering
				\includegraphics[width=\textwidth]{./pictures/superscalar.png}
			\end{figure}
		\end{minipage}		
		
		
		\subsubsection{Pipelining Limitations}
			\begin{table}[H]\centering
				\begin{tabular}{|>{\bfseries}p{0.2\textwidth}|p{0.75\textwidth}|}
					\hline
					Resource conflicts:
						& \textbf{Occur when} two or more resources compete for the same resource at the same time. Superscalar architectures try to resolve resource conflicts by introducing multiple pipelines in parallel.\\
					\hline
					Control (procedural) dependency:
						& \textbf{Occurs because} of branch instructions. In order to identify the next instruction, the current one has to be decoded first.\\
					\hline
					Data conflicts:
						& \textbf{Occur because} of data dependencies between instructions.\newline
							\textbf{True data dependency:} output of one instruction is required as an input to another instruction.\newline
							\textbf{Output dependency:} two instructions are writing to the same location.\newline
							\textbf{Antidependency:} instruction uses a location an operand while a following one is writing into that location.\\
					\hline
				\end{tabular}
				\caption{Pipelining execution limitations}
				\label{tab:pipeliningLimitations}
			\end{table}
			
		\subsubsection{Execution Policies}
			\begin{longtable}{|>{\bfseries}p{0.2\textwidth}|p{0.4\textwidth}|p{0.35\textwidth}|}
				\hline
				Example code:
					&
					& \vspace{0pt}
					
						\includegraphics[width=0.35\textwidth]{./pictures/exampleCodeExecutionPolicies.png}\\
				\hline
				In-order issue,\newline in-order completion
					& Instruction are issued and completed in the same order as they would be in sequential execution. 
					& \vspace{0pt}
					
						\includegraphics[width=0.35\textwidth]{./pictures/policyInOrderInOrder.png}\\
				\hline
				In-order issue,\newline out-of-order completion
					& Instruction are issued in the same order as they would be in sequential execution, but may be completed in a different order.\newline
						\textbf{In case of a conflict} no new instruction can be issued until it has been resolved.
					& \vspace{0pt}
					
						\includegraphics[width=0.35\textwidth]{./pictures/policyInOrderOutOfOrder.png}\\
				\hline
				Out-of-order issue,\newline out-of-order completion
					& Instruction can be issued and completed in a different order as they would be in sequential execution.\newline
						Using this policy, the compiler has to deal with all of the possible data conflicts described in table \ref{tab:pipeliningLimitations}.
					& \vspace{0pt}
					
						\includegraphics[width=0.35\textwidth]{./pictures/policyOutOfOrderOutOfOrder.png}\\
				\hline
			\end{longtable}
			
		\subsubsection{VLIW Processors}
			\begin{longtable}{|>{\bfseries}p{0.18\textwidth}|p{0.8\textwidth}|}
				\hline
				Very Large\newline Instruction Word
					& Detection of instructions that can be parallelised during compile-time and packs them into one large instruction, and corresponding operations are executed in parallel. Detection is based on global analysis of the programme.\\
				\hline
				Advantages:
					& Simple hardware, number of FUs can be increased without additional sophisticated hardware.\\
				\hline
				Problems:
					& \begin{itemize}
					    \item Large number of registers necessary in order to store all operands and results for FUs.
					    \item High bandwidth necessary for data transport.
					    	E.g.: one instruction with $7$ operations and $24$Bits each \textrightarrow\ $168$Bits per instruction.
					    \item Large code size
					    \item Binary code incompatibility between processor versions.
					\end{itemize}\\
				\hline
			\end{longtable}			
			
		\subsubsection{DSPs}
			\begin{longtable}{|>{\bfseries}p{0.18\textwidth}|p{0.8\textwidth}|}
				\hline
				Digital Signal\newline Processors
					& Designed to execute a large number of arithmetic operations in as short a time as possible.\\
				\hline
				Architecture:
					& Modified Harvard architecture. Shared physical memory bus, but use of cache allows to separate data and code accesses.\\
				\hline
				\multicolumn{2}{|l|}{\textbf{Core functional units:}}\\
				\hline
				.M
					& Multiply and accumulate.\\
				\hline
				.D
					& Data access.\\
				\hline
				.L
					& ALU.\\
				\hline
				.S
					& Shifter/ALU.\\
				\hline
			\end{longtable}
		
			DSPs are in general not compiler friendly, over $50\%$ of of DSP solutions are programmed on naked hardware in assembler.
		
		\subsubsection{Software Pipelining}
			The technique of scheduling instructions across several iterations of a loop. Exploits instruction level parallelism on superscalar and VLIW machines. Iterations are overlaid so that an iteration starts before the previous one has completed.
			
		\subsubsection{Pipelining Functional Units}
			By decreasing the clock cycle and scheduling operations across the clock cycles it is possible to save resources, although it might be that this increases the overall execution time.
			 
			\begin{figure}[H]\centering
				\includegraphics[width=0.5\textwidth]{./pictures/pipeliningFunctionalUnits.png}
				\caption{Pipelining Functional Units}
				\label{fig:pipeliningFU}
			\end{figure}
			
			
		\subsubsection{High Level Synthesis}
			High level synthesis is the synthesis of a high-level source code (such as C) into 