\section{Tightly Coupled / Scratchpad Memory \weekDoran{3}}
	Scratchpad memory is basically a RAM block in the same order of magnitude as the cache in both size and speed. Contrary to cache memory it is mapped into the processor's memory space.
	
	\begin{itemize}
	  \item No discrimination between code and data.
	  \item Placement either at run-time or compile time.
	\end{itemize}
	
	\subsection{Static Partitioning \weekPageDoran{3}{47}}
		Memory is divided in different blocks (partitions). The size of the process and the size of the partition is never equal, this causes \textbf{internal fragments}. This can be reduced by making unequal partition sizes and queues for usage of these fragments.
	
	\subsection{Dynamic Partitioning \weekPageDoran{3}{48}}
		Dynamic partitioning creates partition with the exact same size as the processes. This solves the issue of internal fragments, but causes \textbf{external fragments}.
		
		\subsubsection{Placement Algorithms}
			\begin{table}[H]
				\centering
				\begin{tabular}{|p{0.3\linewidth}|p{0.65\linewidth}|}
					\hline
					\textbf{First Fit}
						& Scan memory from top and take first fit. Tends tobe the simplest and produces less fragmentation than \textbf{Next Fit}.\\
					\hline
					\textbf{Next Fit}
						& Scan memory from last allocation and take next best fit.\\
					\hline
					\textbf{Best Fit}
						& Scan entire memory and find best fit out of all possibilities. Causes lots of little fragments.\\
					\hline	
				\end{tabular}
				\caption{Dynamic Partitioning Placement Algorithms}
			\end{table}		
			
	\subsection{Paging \weekPageDoran{3}{49}}
		Process image is divided into pages of a specific size and divide available memory into frames of equal size. Management is done by a \textbf{page table} where entries are the frame numbers and the reference is the page number. 
		
		\begin{figure}[H]\centering
			\includegraphics[scale=0.8]{./pictures/pagingAddressTrans.png}
		\end{figure}	
		